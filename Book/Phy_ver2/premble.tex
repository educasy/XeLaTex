%\usepackage{mathpazo}
\usepackage{amsmath} % จะใช้ package ของ math อื่นใด ให้เรียกก่อน fontspec
\usepackage{amssymb} % จะใช้ package ของ math อื่นใด ให้เรียกก่อน fontspec
\usepackage{amsfonts} % จะใช้ package ของ math อื่นใด ให้เรียกก่อน fontspec
\usepackage{mathspec} % เรียกใช้แค่นี้มีค่า = \usepackage[no-math]{fontspec}\usepackage{mathspec}
\usepackage{xunicode,xltxtra}
\XeTeXlinebreaklocale "th"
\XeTeXlinebreakskip = 0pt plus 1pt %
\defaultfontfeatures{Scale=1.23}
\renewcommand{\baselinestretch}{1.2}
\setmainfont{TH Sarabun New}
\newfontfamily\kodchasal{TH Kodchasal} % ตั้งชื่อฟอนต์ใหม่เพื่อให้ง่ายต่อการใช้งาน เผื่อว่าในเอกสารต้องการให้มีหลายฟอนต์ เวลาใช้ก็ {\examplefont ข้อความต่าง ๆ}
\newfontfamily\niramit{TH Niramit AS}
\newfontfamily\droid{Droid Serif}
%\usepackage[xetex,cmyk,x11names,svgnames,dvips*]{xcolor}
%\usepackage[xetex,rgb,x11names,svgnames,dvips]{xcolor} % -- dvips,svgnames มีชื่อสีซ้ำกัน เลือกอันใดอันหนึ่ง
\usepackage[xetex,cmyk,x11names,svgnames]{xcolor}
\definecolor{MyColor}{rgb}{0.3,0.4,0.5} % กำหนดสี และชื่อที่จะใช้เรียกสีโดย xcolor
\everymath{\displaystyle} % บังคับให้ทุกสมการเป็น displaystyle
% --------------- เปลี่ยน Chapter --> บทที่ ,วันที่ภาษาไทย -------------------
\usepackage{polyglossia}
\setdefaultlanguage{thai}
\newfontfamily{\thaifont}[Script=thai]{TH Sarabun New}
% ---------------------------------------------------------
% --------- Dummy Text -----------
\usepackage{lipsum}
\usepackage{blindtext}
% --------- End Dummy Text -------
\usepackage{marginnote}
\usepackage{paralist}
\usepackage{enumitem,xcolor} % ใช้ rusume ได้ ไม่ต้อง setcounter
\usepackage{tikz}

\usetikzlibrary{shadings}
\usetikzlibrary{decorations.text}
\usepackage{tabls}
\usepackage{graphicx}
\usepackage{array}
\usepackage{tabularx}
\usepackage{booktabs}
\usepackage{longtable}
\usepackage{wrapfig} %ตัวหนังสือล้อมรอบรูปหรือตารางได้ (wrapfigure,wraptable) *** ไม่สามารถอยู่ในพวก enumerate ได้
%\usepackage[Glenn]{fncychap}

% ----------------- Font พวก chapter,section -----------------------------
\usepackage{sectsty,ulem}
%\allsectionsfont{\ulemheading{\uline}} % -- ชื่อ section ขีดเส้นใต้ -----------
\sectionfont{\color{NavyBlue}}
\subsectionfont{\color{Indigo}}


% ---------------------- print the section numbering in the margins -----------------------------
%\makeatletter
%\def\@seccntformat#1{\csname the#1\endcsname)\quad} 
%\makeatother
% -----------------------------------------------------------------------------------------------

\usepackage{eso-pic,picture}

\usepackage{cellspace}
\usepackage[usenames,dvipsnames]{pstricks} 
\usepackage{epsfig} 
\usepackage{pst-grad} % For gradients
\usepackage{pst-plot} % For axes 
\usepackage{makecell}
\usepackage{fancyhdr}
\usepackage{lastpage}
\usepackage{fancybox}
\usepackage{multirow}
\usepackage{calc}
\usepackage{multicol}
\usepackage{multienum}
\usepackage{framed}
%--------------------- กล่องข้อความสวย ๆ ----------------
\usepackage{tcolorbox}
\tcbuselibrary{documentation,skins}
% -----------------------------------------------------
%\usepackage[top=1in,bottom=1in,outer=2in,inner=1.2in,heightrounded,marginparwidth=1in,marginparsep=.7in,showframe]{geometry}
\usepackage[bottom=5cm]{geometry}
% --------- แก้ปัญหา Command \clipbox already defined. หลังจากเพิ่ม adjustbox
% http://www.tex.ac.uk/cgi-bin/texfaq2html?label=alreadydef
\usepackage{savesym}
\savesymbol{clipbox}
% ----------------------------------------------------------------------
\usepackage{adjustbox}
%\restoresymbol{clipbox}

\usepackage{etoolbox}

% ระยะช่องว่างหน้า+หลังเลขหรืออักษรในสมการ หน่วยเป็น mmu , 1 mmu = 1mu/1000 , 18mu = 1 em ,default = 500 mmu = 1/36 em 
% ตัวไหนจะให้มีระยะพิเศษก็ใส่ " นำหน้า เช่น $ x^{"2} หรือใส่ทีละคำให้ใส่เป็น $ x^{\"3yz"} $
% \setminwhitespace[XXXX] 
\setminwhitespace[3000] 

% เลือก Math Font ต่าง ๆ นา ๆ
%\setmathsfont(Digits,Latin){Asana Math}
%\setmathsfont(Digits,Latin){jsMath-cmr10}
%\setmathsfont(Digits,Latin){Kerkis}
%\setmathsfont(Digits,Latin){Neo Euler}
%\setmathsfont(Digits,Latin){Fontin}
%\setmathsfont(Digits,Latin){Plakken}
%\setmathsfont(Digits,Latin){DejaVu Serif}
%\setmathsfont(Digits,Latin){STIXGeneral}
%\setmathsfont(Digits,Latin){CMU Bright}
%\setmathsfont(Digits,Latin){Iwona Light}
%\setmathsfont(Digits,Latin,Greek)[Numbers={Lining,Proportional}]{Iwona Light}
%\setmathsfont(Digits,Latin,Greek){TH Sarabun New}
%\setmathsfont(Digits,Latin,Greek){Mathmos Original}
% ใช้ฟอนต์ OTF ในเอกสารแต่ MATH ใช้ฟอนต์ Math
%\usepackage[no-math]{fontspec}

% ใช้ฟอนต์ OTF ในเอกสารและ MATH ใช้ฟอนต์ OTF
%\usepackage[no-math]{fontspec}
%\usepackage{mathspec}
%\setmainfont{TH Sarabun New}
%\setallmainfonts(Digits,Latin,Greek){TH Sarabun New}
\setallmainfonts(Digits,Latin,Greek){TH Sarabun New}

% \setmathrm จะเปลี่ยนฟอนต์เฉพาะที่อยู่ในคำสั่ง \mathrm{.....} เท่านั้น
%\setmathrm{TH Sarabun New}
%\setmathsfont(Digits,Latin,Greek){TH Sarabun New}
%\setmathfont(Digits,Latin,Greek){TH Sarabun New}

\newcommand*\circled[1]{\tikz[baseline=(char.base)]{%
            \node[shape=circle,fill=blue!20,inner sep=2pt] (char) {#1};}}

%\setlist[enumerate,1]{leftmargin=*,resume}% ตั้งให้ enumerate level 1 ไม่ indent และต่อข้ออัตโนมัติ โดยไม่ต้องใช้ counter
%\setlist[enumerate,1]{leftmargin=*,resume,label=\color{blue}\theenumi}% ใส่สีให้ด้วย
\setlist[enumerate,1]{leftmargin=*,resume,label=\protect\circled{\arabic*}} % วงกลมที่ตัวเลข enumerate

\newcommand{\nonet}{\textbf{(แนว O-NET) }}
\newcommand{\ncmu}{\textbf{(แนว มช.) }}
\newcommand{\cmu}[1]{\textbf{(มช. #1)}}

% ------------------------------------ โจทย์ + รูป + Choice environment ---------------------------------------------------------------------------------------------------
\newenvironment{ljrp} { \begin{minipage}[t]{.65\linewidth} } { \end{minipage} }
\newenvironment{ljrp2} { \begin{minipage}[t]{.5\linewidth} } { \end{minipage} }
\newenvironment{rp}[1] { \hfill \begin{adjustbox}{valign=t} \begin{minipage}[t]{.3\linewidth} \includegraphics[width=\textwidth]{#1} } { \end{minipage} \end{adjustbox} }
\newenvironment{rp2}[1] { \hfill \begin{adjustbox}{valign=t} \begin{minipage}[t]{.45\linewidth} \includegraphics[width=\textwidth]{#1} } { \end{minipage} \end{adjustbox} }
\newenvironment{rjlp} { \hfill \begin{minipage}[t]{.65\linewidth} } { \end{minipage} }
\newenvironment{lp}[1] { \begin{adjustbox}{valign=t} \begin{minipage}[t]{.3\linewidth} \includegraphics[width=\textwidth]{#1} } { \end{minipage} \end{adjustbox} }
% เพิ่ม vspace{-5pt} ก่อน mitemxxx เพื่อลดขนาดบรรทัด
% หรือเพิ่มก่อน choice environment
\newenvironment{4c}[4] { \begin{multienumerate} \setlength{\labelwidth}{8pt} \mitemxxxx{#1}{#2}{#3}{#4}} { \end{multienumerate}}
\newenvironment{2c}[4] { \begin{multienumerate} \setlength{\labelwidth}{8pt} \setlength{\itemsep}{0pt} \mitemxx{#1}{#2} \mitemxx{#3}{#4}} { \end{multienumerate}}
\newenvironment{1c}[4] { 
    \begin{multienumerate} 
    \setlength{\labelwidth}{8pt} 
    \setlength{\itemsep}{0pt} 
    %\setlength{\parsep}{0pt} 
    %\setlength{\topsep}{0pt} 
    %\setlength{\partopsep}{0pt} 
    %\setlength{\parskip}{0pt} 
    \mitemx{#1} \mitemx{#2} \mitemx{#3} \mitemx{#4}
} { \end{multienumerate} }
% -----------------------------------------------------------------------------------------------------------------------------------------------------------------------

%%%%% ----------- พิมพ์รหีสท้ายข้อ ----------------------------------------------
\newcommand\mybox[2][]{\tikz[overlay]\node[fill=blue!20,inner sep=2pt, anchor=text, rectangle, rounded corners=1mm,#1] {#2};\phantom{#2}}
\usepackage{fmtcount}
\usepackage{xifthen,changepage}
\newif\ifrunjc % define a logical variable ชื่อ runjc
\runjctrue % กำหนดให้ runjc เป็น true --> พิมพ์รหัสที่ข้อ + เนื้อหา
%\runjcfalse % กำหนดให้ runjc เป็น false --> ไม่พิมพ์รหัสที่ข้อ + เนื้อหา
\newcounter{runj} % กำหนด counter ใหม่สำหรับ jote ชื่อ runj
\setcounter{runj}{1} % กำหนดค่าเริ่มต้น counter
\newcounter{runc} % กำหนด counter ใหม่สำหรับ content ชื่อ runc
\setcounter{runc}{1} % กำหนดค่าเริ่มต้น counter
\newcommand{\runningj}{% กำหนดคำสั่งพิมพ์รหัสท้ายโจทย์ชื่อ \runningj
    \ifrunjc
        \mybox{P1\padzeroes[4]{\decimal{runj}}}\stepcounter{runj}
        %\hfill \mybox{P1\padzeroes[4]{\decimal{runj}}}\stepcounter{runj}
        % \hfill \mybox[fill=red!20]{p1\padzeroes[4]{\decimal{runj}}}\stepcounter{runj} % เปลี่ยนสีเป็นชมพู
    \fi
}
\newcommand{\runningc}{% กำหนดคำสั่งพิมพ์รหัสท้ายเนื้อหาชื่อ \runningc
    \ifrunjc
        % \ovalbox{p2\padzeroes[4]{\decimal{runc}}}\stepcounter{runc}
        % \hfill\mybox{\textbf{P2\padzeroes[4]{\decimal{runc}}}}\stepcounter{runc}
        \noindent $\triangleright\triangleright$ \textbf{P2\padzeroes[4]{\decimal{runc}}}\stepcounter{runc}
    \fi
}
%%%% ----------------------------------------------------------------------

% -------- Environment for Content --------------------
\newenvironment{c1}%
{\begin{Sbox}\begin{minipage}}%
{\end{minipage}\end{Sbox}\begin{center}\fbox{\TheSbox}\end{center}}
%%%%%% -Usage-  \begin{c1}{.9\textwidth}            กรณีที่วัตถุเคลื่อนที่ไปด้วยความเร็วคงที่   จะได้ว่า 
            \begin{align*} % ให้เครื่องหมาย = ในสมการเท่ากัน และไม่มีเลขกำกับสมการ
                \text{ระยะทางที่เคลื่อนที่ได้}  &=  \text{อัตราเร็ว}\times \text{เวลาที่ใช้เคลื่อนที่} \\
                \text{\textbf{หรือ}}\qquad \qquad              s  &=  v \cdot  t
            \end{align*}
            \begin{tabbing}
                    \textbf{เมื่อ} \quad  \=\textbf{s}  คือระยะทางที่เคลื่อนที่ได้ \quad \=หน่วยเป็นเมตร ( m ) \\
                                        \>\textbf{v}  คืออัตราเร็วซึ่งคงที่  \> หน่วยเป็นเมตรต่อวินาที ( m/s ) \\
                                        \>\textbf{t}  คือเวลาที่ใช้เคลื่อนที่  \> หน่วยเป็นวินาที ( s ) 
            \end{tabbing}
\end{c1} %%%%%%%
\newenvironment{c2}%
{\begin{tcolorbox}[title=\runningc,colback=white]}
{\end{tcolorbox}}
\newenvironment{c3}%
%{\begin{tcolorbox}[title=\runningc,skin=widget,colback=Wheat!50!white,colframe=FireBrick!75!black]}
{\begin{tcolorbox}[title=\runningc,colback=LightYellow1,colframe=OrangeRed4]}
{\end{tcolorbox}}
%%%%%% -Usage-  \begin{c2}            กรณีที่วัตถุเคลื่อนที่ไปด้วยความเร็วคงที่   จะได้ว่า 
            \begin{align*} % ให้เครื่องหมาย = ในสมการเท่ากัน และไม่มีเลขกำกับสมการ
                \text{ระยะทางที่เคลื่อนที่ได้}  &=  \text{อัตราเร็ว}\times \text{เวลาที่ใช้เคลื่อนที่} \\
                \text{\textbf{หรือ}}\qquad \qquad              s  &=  v \cdot  t
            \end{align*}
            \begin{tabbing}
                    \textbf{เมื่อ} \quad  \=\textbf{s}  คือระยะทางที่เคลื่อนที่ได้ \quad \=หน่วยเป็นเมตร ( m ) \\
                                        \>\textbf{v}  คืออัตราเร็วซึ่งคงที่  \> หน่วยเป็นเมตรต่อวินาที ( m/s ) \\
                                        \>\textbf{t}  คือเวลาที่ใช้เคลื่อนที่  \> หน่วยเป็นวินาที ( s ) 
            \end{tabbing}
\end{c2} %%%%%%%

% -------- Environment for Jote ----------------------
%\newenvironment{jote}
%{\begin{enumerate}[resume] \item \runningj}
%{\end{enumerate}}

%\usepackage{anyfontsize}
\usepackage{lmodern}  % ปรับขนาด font ด้วย {\fontsize{52}{58} TEXT}
\usepackage{moresize} % เพิ่ม \HUGE , \ssmall
\def\CHshift#1{\raisebox{2pt}}



\newfontfamily\nanum{NanumGothic}
\newfontfamily\maven{Maven Pro Light}
\newfontfamily\blanch{Blanch Condensed}
\newfontfamily\blanchI{Blanch Condensed Inline}
\newfontfamily\blanchCI{Blanch Caps Light}
\newfontfamily\blanchL{Blanch Condensed Light}
\newfontfamily\blanchC{Blanch Caps}
\newfontfamily\blanchCI{Blanch Caps Inline}
\newfontfamily\codeL{Code Light}
\newfontfamily\codeB{Code Bold}
\newfontfamily\fabrica{Fabrica}
\newfontfamily\scifly{SciFly}
\newfontfamily\sket{Sketchetik}
\newfontfamily\akhanake{Book_Akhanake}
\newfontfamily\sarang{Layiji SaRangHeYo OT}
\newfontfamily\sig{Signika}
\newfontfamily\quark{Quark}
\newfontfamily\fontcraft{Fontcraft}

