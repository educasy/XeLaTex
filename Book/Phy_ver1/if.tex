\documentclass[a4paper,12pt]{book}
%\usepackage{mathpazo}
\usepackage{amsmath} % จะใช้ package ของ math อื่นใด ให้เรียกก่อน fontspec
\usepackage{amssymb} % จะใช้ package ของ math อื่นใด ให้เรียกก่อน fontspec
\usepackage{amsfonts} % จะใช้ package ของ math อื่นใด ให้เรียกก่อน fontspec
\usepackage{mathspec} % เรียกใช้แค่นี้มีค่า = \usepackage[no-math]{fontspec}\usepackage{mathspec}
\usepackage{xunicode,xltxtra}
\XeTeXlinebreaklocale "th"
\XeTeXlinebreakskip = 0pt plus 1pt %
\defaultfontfeatures{Scale=1.23}
\renewcommand{\baselinestretch}{1.2}
\setmainfont{TH Sarabun New}
\newfontfamily\kodchasal{TH Kodchasal} % ตั้งชื่อฟอนต์ใหม่เพื่อให้ง่ายต่อการใช้งาน เผื่อว่าในเอกสารต้องการให้มีหลายฟอนต์ เวลาใช้ก็ {\examplefont ข้อความต่าง ๆ}
\newfontfamily\niramit{TH Niramit AS}
\usepackage{xcolor}
\definecolor{MyColor}{rgb}{0.3,0.4,0.5} % กำหนดสี และชื่อที่จะใช้เรียกสีโดย xcolor
\everymath{\displaystyle} % บังคับให้ทุกสมการเป็น displaystyle
\usepackage{tabls}
\usepackage{graphicx}
\usepackage{tabularx}
\usepackage{booktabs}
\usepackage{longtable}
\usepackage{wrapfig} %ตัวหนังสือล้อมรอบรูปหรือตารางได้ (wrapfigure,wraptable) *** ไม่สามารถอยู่ในพวก enumerate ได้
\usepackage[Glenn]{fncychap}
\usepackage{sectsty,ulem}
\allsectionsfont{\ulemheading{\uline}}
\makeatletter
\def\@seccntformat#1{\csname the#1\endcsname)\quad}
\makeatother
\usepackage{cellspace}
\usepackage[usenames,dvipsnames]{pstricks} 
\usepackage{epsfig} 
\usepackage{pst-grad} % For gradients
\usepackage{pst-plot} % For axes 
\usepackage{makecell}
\usepackage{fancyhdr}
\usepackage{lastpage}
\usepackage{fancybox}
\usepackage{multirow}
\usepackage{calc}
% ระยะช่องว่างหน้า+หลังเลขหรืออักษรในสมการ หน่วยเป็น mmu , 1 mmu = 1mu/1000 , 18mu = 1 em ,default = 500 mmu = 1/36 em 
% ตัวไหนจะให้มีระยะพิเศษก็ใส่ " นำหน้า เช่น $ x^{"2} หรือใส่ทีละคำให้ใส่เป็น $ x^{\"3yz"} $
% \setminwhitespace[XXXX] 
\setminwhitespace[3000] 

% เลือก Math Font ต่าง ๆ นา ๆ
%\setmathsfont(Digits,Latin){Asana Math}
%\setmathsfont(Digits,Latin){jsMath-cmr10}
%\setmathsfont(Digits,Latin){Kerkis}
%\setmathsfont(Digits,Latin){Neo Euler}
%\setmathsfont(Digits,Latin){Fontin}
%\setmathsfont(Digits,Latin){Plakken}
%\setmathsfont(Digits,Latin){DejaVu Serif}
%\setmathsfont(Digits,Latin){STIXGeneral}
%\setmathsfont(Digits,Latin){CMU Bright}
%\setmathsfont(Digits,Latin){Iwona Light}
%\setmathsfont(Digits,Latin,Greek)[Numbers={Lining,Proportional}]{Iwona Light}
%\setmathsfont(Digits,Latin,Greek){TH Sarabun New}
%\setmathsfont(Digits,Latin,Greek){Mathmos Original}
% ใช้ฟอนต์ OTF ในเอกสารแต่ MATH ใช้ฟอนต์ Math
%\usepackage[no-math]{fontspec}

% ใช้ฟอนต์ OTF ในเอกสารและ MATH ใช้ฟอนต์ OTF
%\usepackage[no-math]{fontspec}
%\usepackage{mathspec}
%\setmainfont{TH Sarabun New}
%\setallmainfonts(Digits,Latin,Greek){TH Sarabun New}
\setallmainfonts(Digits,Latin,Greek){TH Sarabun New}

% \setmathrm จะเปลี่ยนฟอนต์เฉพาะที่อยู่ในคำสั่ง \mathrm{.....} เท่านั้น
%\setmathrm{TH Sarabun New}
%\setmathsfont(Digits,Latin,Greek){TH Sarabun New}
%\setmathfont(Digits,Latin,Greek){TH Sarabun New}

%\usepackage{calculator}
\usepackage{fmtcount}
\usepackage{xifthen}
\begin{document}

%\newcommand{\isitthree}[1] { \ifnum#1=3 number #1 is 3 \else number #1 is not 3 \fi }
%We know that \isitthree{33}

%\newcounter{mycounter}%
%\newcommand{\printcntr}{%
%  \stepcounter{mycounter}%
%  p1\padzeroes[4]{\decimal{mycounter}}% 
%}

%\printcntr

%\newcounter{tt}
%\newcommand{\pp}{
%	\ifthenelse{\themycounter<10}
%	{p1\padzeroes[4]{\decimal{tt}}\\}
%	{NNNNNN\\}
%	\stepcounter{tt}
%}

\newif\ifrunjc % define a logical variable ชื่อ runjc
\runjctrue % กำหนดให้ runjc เป็น true --> พิมพ์รหัสที่ข้อ + เนื้อหา
%\runjcfalse % กำหนดให้ runjc เป็น false --> ไม่พิมพ์รหัสที่ข้อ + เนื้อหา

\newcounter{runj} % กำหนด counter ใหม่สำหรับ jote ชื่อ runj
\setcounter{runj}{1} % กำหนดค่าเริ่มต้น counter
\newcounter{runc} % กำหนด counter ใหม่สำหรับ content ชื่อ runc
\setcounter{runc}{1} % กำหนดค่าเริ่มต้น counter
\newcommand{\runningj}{% กำหนดคำสั่งพิมพ์รหัสท้ายโจทย์ชื่อ \runningj
	\ifrunjc
		\ovalbox{p1\padzeroes[4]{\decimal{runj}}} \\ \stepcounter{runj}
	\fi
}
\newcommand{\runningc}{% กำหนดคำสั่งพิมพ์รหัสท้ายเนื้อหาชื่อ \runningc
	\ifrunjc
		\ovalbox{p2\padzeroes[4]{\decimal{runc}}} \\ \stepcounter{runc}
	\fi
}

\noindent
\runningj
\runningj
\runningj
\runningj
\runningj
\runningj
\runningj
\runningj
\runningj
\runningj
\runningj
\runningj
\runningj
\runningj
\runningc
\runningc
\runningc
\runningc
\runningc
\runningc
\runningc
\runningc
\runningc

%\newcounter{mycount}
%\noindent\whiledo{\themycount<10}{%
%  \ifthenelse{\isodd{\thepage}}%
%    {Odd\\}%
%    {Even\\}%
%  \stepcounter{mycount}%
%}


%\newcommand{\running}{\textbf{XXX}}

%\COPY{1}{\StartVal}
%\COPY{\StartVal}{\ActVal}
%\COPY{0}{\TempVal}
%ActVal is ---- \ActVal ----- \\
%TempVal is ---- \TempVal ----- \\
%%xxxxxxxxxx p1000\StartVal
%\ADD{1}{\TempVal}{\TempVal}
%\COPY{\res}{\TempVal}
%\ifthenelse{\equal{\TempVal}{0}}{Yessss}{Nooooo}\\
%TempVal is ---- \TempVal ----- \\
%\running \\
%p1\padzeroes[3]\StartVal

%\newcommand{\zzz}{ 
%\ifnum\StartVal<10 
%	\ADD{1}{\StartVal}{\TTT}
%\fi
%	p1\padzeroes[4]\TTT
	%StartVal is ---- \StartVal ----- \\
%}
%	p1\padzeroes[4]\StartVal
%\zzz
%\zzz
%\zzz
%\zzz
%\zzz
%\zzz
%\zzz
\end{document}
